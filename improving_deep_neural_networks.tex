\ifx\PREAMBLE\undefined
\documentclass{report}
\usepackage[format = hang, font = bf]{caption}
\usepackage{graphicx}
\usepackage{array}
\usepackage{amsmath}
\usepackage{colortbl}
\usepackage{empheq}
\usepackage{mathtools}
\usepackage{boxedminipage}
\usepackage{listings}
\usepackage{makecell}%diagonal line in table
\usepackage{float}%allowing forceful figure[H]
\usepackage{xcolor}
\usepackage{amsfonts}%allowing \mathbb{R}
\usepackage{amssymb}
\usepackage{bm}
\usepackage{alltt}
\usepackage{amsthm}
\usepackage[super]{nth}
\usepackage{tikz}
\usetikzlibrary{positioning, calc}
\usepackage{multirow}
\usepackage{algorithmicx}
\usepackage[chapter]{algorithm} 
%chapter option ensures that algorithms are numbered within each chapter rather than in the whole article
\usepackage[noend]{algpseudocode} %If end if, end procedure, etc is expected to appear, remove the noend option
\usepackage{xspace}
\usepackage{color}
\usepackage{url}
\def\UrlBreaks{\do\A\do\B\do\C\do\D\do\E\do\F\do\G\do\H\do\I\do\J\do\K\do\L\do\M\do\N\do\O\do\P\do\Q\do\R\do\S\do\T\do\U\do\V\do\W\do\X\do\Y\do\Z\do\[\do\\\do\]\do\^\do\_\do\`\do\a\do\b\do\c\do\d\do\e\do\f\do\g\do\h\do\i\do\j\do\k\do\l\do\m\do\n\do\o\do\p\do\q\do\r\do\s\do\t\do\u\do\v\do\w\do\x\do\y\do\z\do\0\do\1\do\2\do\3\do\4\do\5\do\6\do\7\do\8\do\9\do\.\do\@\do\\\do\/\do\!\do\_\do\|\do\;\do\>\do\]\do\)\do\,\do\?\do\'\do+\do\=\do\#\do\-}
\usepackage[breaklinks = true]{hyperref}
\lstset{language = c++, breaklines = true, tabsize = 2, numbers = left, extendedchars = false, basicstyle = {\ttfamily \footnotesize}, keywordstyle=\color{blue!70}, commentstyle=\color{red!70}, frame=shadowbox, rulesepcolor=\color{red!20!green!20!blue!20}, numberstyle={\color[RGB]{0,192,192}}}
\mathchardef\myhyphen="2D
% switch-case environment definitions
\algblock{switch}{endswitch} 
\algblock{case}{endcase}
%\algrenewtext{endswitch}{\textbf{end switch}} %If end switch is expected to appear, uncomment this line.
\algtext*{endswitch} % Make end switch disappear
\algtext*{endcase}
\allowdisplaybreaks
\DeclareMathOperator*{\argmin}{\arg\!\min}
\DeclareMathOperator*{\argmax}{\arg\!\max}

\begin{document}
\fi
\chapter{Improving Deep Neural Networks}
\section{Machine Learning Application Setup}
\subsection{Train/Dev/Test Sets}
\begin{itemize}
  \item Applying ML to practical problems is a highly iterative problem. Hyper parameters can only be obtained by looping through: idea $\Rightarrow$ code $\Rightarrow$ experiment.
  \item Data should be divided into training set, hold-out cross validation set (dev set) and testing set. For small dataset(size $<10^6$): 60/20/20. For large dataset(size > $10^6$): 98/1/1(i.e. most data in training set). Build a model on training set $\Rightarrow$ optimize hyper parameters on dev set $\Rightarrow$ evaluate model on testing set. 
  \item It should be guaranteed that dev set and testing set come from the same distribution.
  \item Sometimes it's ok not to have a testing set (no model evaluation)
\end{itemize}
\subsection{Bias and Variance}
\begin{itemize}
  \item High bias: underfitting. High variance: overfitting
  \item Tell bias \& variance from errors:
    \begin{itemize}
      \item low train error$(1\%)$ \& high dev error$(15\%)$: high variance
      \item high train error$(15\%)$ \& high dev error$(15\%)$: high bias
      \item high train error$(15\%)$ \& higher dev error$(15\%)$: high bias \& high variance
      \item low train error$(1\%)$ \& low dev error$(1\%)$: best situation 
    \end{itemize}
  \item The above method is based on the fact that human can solve the problem perfectly with 0\% error. If the intrinsic complexity of the problem causes non-zero human error, it should be used as the baseline. 
  \item Solution to high bias: bigger NN(more neurons in each layer, more layers), try different model suitable for the dataset, run the algorithm longer (more iterations), advanced Optimization methods, etc
  \item Solution to high variance: obtain more data, regularization, try different model suitabloe for the dataset, etc
  \item Bigger NN never hurts
\end{itemize}
\section{Regularization}
Adding regularization to NN helps to reduce overfitting (high variance)
\subsection{L2 Regularization}
\begin{itemize}
  \item L2 norm is called Frobenius norm
  \item L1 regularization makes the model sparse (i.e. many 0 in $\mathbf{w}$. L2 regularization of logistic regression is often used:
  \[\mathcal{J}(\mathbf{w},b)=\frac{1}{m}\displaystyle\sum_{i=1}^m\mathcal{L}(\hat{y}^{(i)}, y^{(i)})+\frac{\lambda}{2m}\vert\vert\mathbf{w}\vert\vert^2_2\] 
  \item L2 regularization of NN: 
  \[\mathcal{J}(\mathbf{W}^{[1]},\mathbf{b}^{[1]},\cdots,\mathbf{W}^{[L]},\mathbf{b}^{[L]})=\frac{1}{m}\displaystyle\sum_{i=1}^m\mathcal{L}(\hat{y}^{(i)}, y^{(i)})+\frac{\lambda}{2m}\displaystyle\sum_{i=1}^L\vert\vert\mathbf{W}^{[l]}\vert\vert^2_F\] 
  \item Back propagation \& parameter update of NN with L2 regularization
  \begin{align*}
  \frac{\partial \mathcal{J}}{\partial \mathbf{W}^{[l]}}&=\textbf{Result without regularization}+\frac{\lambda}{m}\mathbf{W}^{[l]}\\
  \mathbf{W}^{[l]}&\coloneqq\left(1-\frac{\alpha\lambda}{m}\right)\mathbf{W}^{[l]}-\alpha(\textbf{Result without regularization})\\
  \end{align*}
  Obviously the norm of the weights shrink at each iteration, called weight decay. 
\end{itemize}
\subsection{Dropout Regularization}
\section{Optimization Problem Setup}
\ifx\PREAMBLE\undefined
\end{document}
\fi