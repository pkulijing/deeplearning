\ifx\PREAMBLE\undefined
\documentclass{report}
\usepackage[format = hang, font = bf]{caption}
\usepackage{graphicx}
\usepackage{array}
\usepackage{amsmath}
\usepackage{mathtools}
\usepackage{boxedminipage}
\usepackage{listings}
\usepackage{makecell}%diagonal line in table
\usepackage{float}%allowing forceful figure[H]
\usepackage{xcolor}
\usepackage{amsfonts}%allowing \mathbb{R}
\usepackage{alltt}
\usepackage{algorithmicx}
\usepackage[chapter]{algorithm} 
%chapter option ensures that algorithms are numbered within each chapter rather than in the whole article
\usepackage[noend]{algpseudocode} %If end if, end procdeure, etc is expected to appear, remove the noend option
\usepackage{xspace}
\usepackage{color}
\usepackage{url}
\def\UrlBreaks{\do\A\do\B\do\C\do\D\do\E\do\F\do\G\do\H\do\I\do\J\do\K\do\L\do\M\do\N\do\O\do\P\do\Q\do\R\do\S\do\T\do\U\do\V\do\W\do\X\do\Y\do\Z\do\[\do\\\do\]\do\^\do\_\do\`\do\a\do\b\do\c\do\d\do\e\do\f\do\g\do\h\do\i\do\j\do\k\do\l\do\m\do\n\do\o\do\p\do\q\do\r\do\s\do\t\do\u\do\v\do\w\do\x\do\y\do\z\do\0\do\1\do\2\do\3\do\4\do\5\do\6\do\7\do\8\do\9\do\.\do\@\do\\\do\/\do\!\do\_\do\|\do\;\do\>\do\]\do\)\do\,\do\?\do\'\do+\do\=\do\#\do\-}
\usepackage[breaklinks = true]{hyperref}
\lstset{language = c++, breaklines = true, tabsize = 2, numbers = left, extendedchars = false, basicstyle = {\ttfamily \footnotesize}, keywordstyle=\color{blue!70}, commentstyle=\color{red!70}, frame=shadowbox, rulesepcolor=\color{red!20!green!20!blue!20}, numberstyle={\color[RGB]{0,192,192}}}
\mathchardef\myhyphen="2D
% switch-case environment definitions
\algblock{switch}{endswitch} 
\algblock{case}{endcase}
%\algrenewtext{endswitch}{\textbf{end switch}} %If end switch is expected to appear, uncomment this line.
\algtext*{endswitch} % Make end switch disappear
\algtext*{endcase}
\allowdisplaybreaks
\begin{document}
\fi
\chapter{Sequence Models}
Examples of sequence model use cases:
\begin{itemize}
  \item Speech recognition: audio clip $\rightarrow$ transcript
  \item Music generation: genre of music $\rightarrow$ music clip
  \item Sentiment classification: ``there is nothing to like in this movie'' $\rightarrow$ one star
  \item DNA sequence analysis: DNA sequence $\rightarrow$ sequences encoding proteins
  \item Machine translation: French sentence $\rightarrow$ English sentence
  \item Video activity recognition: video of people running $\rightarrow$ running
  \item Named entity recognition: text $\rightarrow$ names in the text
\end{itemize}
Notation (using named entity recognition as an example):
\begin{itemize}
\item Input $x$: \textit{Harry Porter} and \textit{Hermione Granger} invented a new spell. $x^{\langle t\rangle}$ is the $t^{th}$ word.
\item Output $y$ (whether each word is part of a name): 110110000. $y^{\langle t\rangle}$ for the $t^{th}$ word.
\item Length of the sequences $T_x=9, T_y=9$
\item Same notation as before for training example index: $X^{(i)\langle t\rangle}$, $Y^{(i)\langle t\rangle}$, $T_x^{(i)}$, $T_y^{(i)}$.
\item Content of $x^{\langle t\rangle}$: define a vocabulary containing all words. Each word is represented by a one-hot vector (one at the index of the word in the vocabulary) whose dimension is the vocabulary size. The vocabulary contains an \textit{unknown} item to represent words not in the vocabulary.
\end{itemize}
\section{Recurrent Neural Networks}
\subsection{Problem of Standard NN}
\begin{itemize}
  \item Input/output of different training examples have different dimensions.
  \item Features learned across different positions of text are not shared.
  \item Large number of parameters because input size is large.
\end{itemize}
\subsection{RNN Structure}
\begin{center}
  \begin{tabular}{cccccccccc}
    & $\hat{y}^{\langle 1\rangle}$ & & $\hat{y}^{\langle 2\rangle}$ & & $\hat{y}^{\langle 3\rangle}$ & & & & $\hat{y}^{\langle T_y\rangle}$\\
    & $\uparrow$ & & $\uparrow$ & & $\uparrow$ & & & & $\uparrow$\\ 
    $\xrightarrow{a^{\langle 0\rangle}}$& NN & $\xrightarrow{a^{\langle 1\rangle}}$ & NN & $\xrightarrow{a^{\langle 2\rangle}}$ & NN & $\xrightarrow{a^{\langle 3\rangle}}$ & $\cdots$ & $\xrightarrow{a^{\langle T_x-1\rangle}}$ & NN \\ 
    & $\uparrow$ & & $\uparrow$ & & $\uparrow$ & & & & $\uparrow$\\ 
    & $x^{\langle 1\rangle}$ & & $x^{\langle 2\rangle}$ & & $x^{\langle 3\rangle}$ & & & & $x^{\langle T_x\rangle}$\\
  \end{tabular}
\end{center}
\begin{itemize}
  \item The activation for $x^{\langle t\rangle}$ is fed as an input for the next item in the sequence $x^{\langle t+1\rangle}$.\footnote{Later items in the sequence are not used for making predictions. The issue will be solved by BRNN (B for bidirectional).}
  \item Parameters are shared among time steps.
  \item $a^{\langle 0\rangle}$ is a fake time 0 activation, usually all zero.
  \item Here we assume $T_x=T_y$. 
  \end{itemize}
Calculation:
\begin{align*}
a^{\langle t\rangle}&=g_a\left(W_{aa}a^{\langle t-1\rangle}+W_{ax}x^{\langle t\rangle}+b_a\right)=g_a\left(W_{a}\left[a^{\langle t-1\rangle}, x^{\langle t\rangle}\right]+b_a\right)\\
\hat{y}^{\langle t\rangle}&=g_y\left(W_{ya}a^{\langle t\rangle}+b_y\right)=g_y\left(W_{y}a^{\langle t\rangle}+b_y\right)
\end{align*}
\begin{itemize}
\item $W_a\equiv\left[W_{aa}, W_{ax}\right], W_y\equiv W_{ya}$
\item $g_a$ is usually $\tanh$ or Relu.
\item $g_y$ depends on output $\hat{y}$ (e.g. sigmoid for binary output).
\end{itemize}
\subsection{Backpropagation through Time}
Loss function:
\begin{align*}
  \mathcal{L}^{\langle t\rangle}\left(\hat{y}^{\langle t\rangle}, y^{\langle t\rangle}\right)&=-y^{\langle t\rangle}\log\hat{y}^{\langle t\rangle}-\left(1-y^{\langle t\rangle}\right)\log\left(1-\hat{y}^{\langle t\rangle}\right)\\
  \mathcal{L}&=\displaystyle\sum_{t=1}^{T_x}\mathcal{L}^{\langle t\rangle}\left(\hat{y}^{\langle t\rangle}, y^{\langle t\rangle}\right)
\end{align*}
\begin{center}
  \begin{tabular}{cccccccccc}
    & $\mathcal{L}^{\langle 1\rangle}$ & & $\mathcal{L}^{\langle 2\rangle}$ & & $\mathcal{L}^{\langle 3\rangle}$ & & & & $\mathcal{L}^{\langle T_y\rangle}$\\
    & $\uparrow\color{red}\downarrow$ & & $\uparrow\color{red}\downarrow$ & & $\uparrow\color{red}\downarrow$ & & & & $\uparrow\color{red}\downarrow$\\ 
    & $\hat{y}^{\langle 1\rangle}$ & & $\hat{y}^{\langle 2\rangle}$ & & $\hat{y}^{\langle 3\rangle}$ & & & & $\hat{y}^{\langle T_y\rangle}$\\
    & $\uparrow\color{red}\downarrow$ & & $\uparrow\color{red}\downarrow$ & & $\uparrow\color{red}\downarrow$ & & & & $\uparrow\color{red}\downarrow$\\ 
    \makecell{$\xrightarrow{a^{\langle 0\rangle}}$\\\color{red}$\xleftarrow{\hphantom{a^{\langle 0\rangle}}}$}& NN & \makecell{$\xrightarrow{a^{\langle 1\rangle}}$\\\color{red}$\xleftarrow{\hphantom{a^{\langle 0\rangle}}}$} & NN & \makecell{$\xrightarrow{a^{\langle 2\rangle}}$\\\color{red}$\xleftarrow{\hphantom{a^{\langle 2\rangle}}}$} & NN & \makecell{$\xrightarrow{a^{\langle 3\rangle}}$\\\color{red}$\xleftarrow{\hphantom{a^{\langle 0\rangle}}}$} & $\cdots$ & \makecell{$\xrightarrow{a^{\langle T_x-1\rangle}}$\\\color{red}$\xleftarrow{\hphantom{a^{\langle T_x-1\rangle}}}$} & NN \\ 
    & $\uparrow$ & & $\uparrow$ & & $\uparrow$ & & & & $\uparrow$\\ 
    & $x^{\langle 1\rangle}$ & & $x^{\langle 2\rangle}$ & & $x^{\langle 3\rangle}$ & & & & $x^{\langle T_x\rangle}$\\
  \end{tabular}
\end{center}
\subsection{Different Types of RNN}
\subsubsection{Many-to-many}
The RNN we saw above is a many-to-many architecture, satisfying $T_x=T_y$:
\begin{center}
  \begin{tabular}{cccccccccc}
    & $\hat{y}^{\langle 1\rangle}$ & & $\hat{y}^{\langle 2\rangle}$ & & $\hat{y}^{\langle 3\rangle}$ & & & & $\hat{y}^{\langle T_y\rangle}$\\
    & $\uparrow$ & & $\uparrow$ & & $\uparrow$ & & & & $\uparrow$\\ 
    $\xrightarrow{a^{\langle 0\rangle}}$& NN & $\xrightarrow{a^{\langle 1\rangle}}$ & NN & $\xrightarrow{a^{\langle 2\rangle}}$ & NN & $\xrightarrow{a^{\langle 3\rangle}}$ & $\cdots$ & $\xrightarrow{a^{\langle T_x-1\rangle}}$ & NN \\ 
    & $\uparrow$ & & $\uparrow$ & & $\uparrow$ & & & & $\uparrow$\\ 
    & $x^{\langle 1\rangle}$ & & $x^{\langle 2\rangle}$ & & $x^{\langle 3\rangle}$ & & & & $x^{\langle T_x\rangle}$\\
  \end{tabular}
\end{center}
\subsubsection{Many-to-one}
The assumption $T_x=T_y$ may not always hold. For example, sentiment classification is a many-to-one architecture:
\begin{center}
  \begin{tabular}{cccccccccc}
    & & & & & & & & & $\hat{y}$\\
    & & & & & & & & & $\uparrow$\\ 
    $\xrightarrow{a^{\langle 0\rangle}}$& NN & $\xrightarrow{a^{\langle 1\rangle}}$ & NN & $\xrightarrow{a^{\langle 2\rangle}}$ & NN & $\xrightarrow{a^{\langle 3\rangle}}$ & $\cdots$ & $\xrightarrow{a^{\langle T_x-1\rangle}}$ & NN \\ 
    & $\uparrow$ & & $\uparrow$ & & $\uparrow$ & & & & $\uparrow$\\ 
    & $x^{\langle 1\rangle}$ & & $x^{\langle 2\rangle}$ & & $x^{\langle 3\rangle}$ & & & & $x^{\langle T_x\rangle}$\\
  \end{tabular}
\end{center}
\subsubsection{One-to-many}
Music generation is a one-to-many architecture:
\begin{center}
  \begin{tabular}{cccccccccc}
    & $\hat{y}^{\langle 1\rangle}$ & & $\hat{y}^{\langle 2\rangle}$ & & $\hat{y}^{\langle 3\rangle}$ & & & & $\hat{y}^{\langle T_y\rangle}$\\
    & $\uparrow$ & & $\uparrow$ & & $\uparrow$ & & & & $\uparrow$\\ 
    $\xrightarrow{a^{\langle 0\rangle}}$& NN & $\xrightarrow{a^{\langle 1\rangle}}$ & NN & $\xrightarrow{a^{\langle 2\rangle}}$ & NN & $\xrightarrow{a^{\langle 3\rangle}}$ & $\cdots$ & $\xrightarrow{a^{\langle T_x-1\rangle}}$ & NN \\ 
    & $\uparrow$ \\ 
    & $x$ \\
  \end{tabular}
\end{center}
In this architecture, the output $y^{\langle t\rangle}$ is often fed to the $t+1^{th}$ step as an input:
\begin{center}
  \begin{tabular}{cccccccccc}
    & $\hat{y}^{\langle 1\rangle}$ & & $\hat{y}^{\langle 2\rangle}$ & & $\hat{y}^{\langle 3\rangle}$ & & & & $\hat{y}^{\langle T_y\rangle}$\\
    & $\uparrow$ & & $\uparrow$ & & $\uparrow$ & & & & $\uparrow$\\ 
    $\xrightarrow{a^{\langle 0\rangle}}$& NN & $\xrightarrow{a^{\langle 1\rangle}}$ & NN & $\xrightarrow{a^{\langle 2\rangle}}$ & NN & $\xrightarrow{a^{\langle 3\rangle}}$ & $\cdots$ & $\xrightarrow{a^{\langle T_x-1\rangle}}$ & NN \\ 
    & $\uparrow$ & & $\uparrow$ & & $\uparrow$ & & & & $\uparrow$\\ 
    & $x$ & & \color{red}$\hat{y}^{\langle 1\rangle}$ & & \color{red}$\hat{y}^{\langle 2\rangle}$ & & & & \color{red}$\hat{y}^{\langle T_y-1\rangle}$\\
  \end{tabular}
\end{center}
\subsubsection{One-to-one}
One-to-one structure is a simple standard NN(no actual RNN):
\begin{center}
  \begin{tabular}{c}
    $\hat{y}^{\langle 1\rangle}$\\
    $\uparrow$\\ 
    NN\\
    $\uparrow$\\ 
    $x^{\langle 1\rangle}$ 
  \end{tabular}
\end{center}
\subsubsection{Many-to-many(machine translation)}
Machine translation uses a special many-to-many architecture:
\begin{center}
  \begin{tabular}{cccccccc}
    $\xrightarrow{a^{\langle 0\rangle}}$& NN & $\xrightarrow{a^{\langle 1\rangle}}$ & NN & $\xrightarrow{a^{\langle 2\rangle}}$ & $\cdots$ & $\xrightarrow{a^{\langle T_x-1\rangle}}$ & NN \\ 
    & $\uparrow$ & & $\uparrow$ & & & & $\uparrow$ \\
    & $x^{\langle 1\rangle}$ & & $x^{\langle 2\rangle}$ & & & & $x^{\langle T_x\rangle}$ \\\arrayrulecolor{red}\hline
    & $\hat{y}^{\langle 1\rangle}$ & & $\hat{y}^{\langle 2\rangle}$ & & & & $\hat{y}^{\langle T_y\rangle}$\\
    & $\uparrow$ & & $\uparrow$ & & & & $\uparrow$\\
    $\xrightarrow{a^{\langle T_x\rangle}}$ & NN & $\xrightarrow{a^{\langle T_x+1\rangle}}$ & NN & $\xrightarrow{a^{\langle T_x+2\rangle}}$ &  $\cdots$ & $\xrightarrow{a^{\langle T_x+T_y-1\rangle}}$ & NN \\
  \end{tabular}
\end{center}
It comprises an encoder ($1\sim T_x$) and a decoder ($T_x+1\sim T_x+T_y$).
\subsection{Sequence Generation}
\ifx\PREAMBLE\undefined
\end{document}
\fi