\ifx\PREAMBLE\undefined
\documentclass{report}
\usepackage[format = hang, font = bf]{caption}
\usepackage{graphicx}
\usepackage{array}
\usepackage{amsmath}
\usepackage{colortbl}
\usepackage{empheq}
\usepackage{mathtools}
\usepackage{boxedminipage}
\usepackage{listings}
\usepackage{makecell}%diagonal line in table
\usepackage{float}%allowing forceful figure[H]
\usepackage{xcolor}
\usepackage{amsfonts}%allowing \mathbb{R}
\usepackage{amssymb}
\usepackage{bm}
\usepackage{alltt}
\usepackage{amsthm}
\usepackage[super]{nth}
\usepackage{tikz}
\usetikzlibrary{positioning, calc}
\usepackage{multirow}
\usepackage{algorithmicx}
\usepackage[chapter]{algorithm} 
%chapter option ensures that algorithms are numbered within each chapter rather than in the whole article
\usepackage[noend]{algpseudocode} %If end if, end procedure, etc is expected to appear, remove the noend option
\usepackage{xspace}
\usepackage{color}
\usepackage{url}
\def\UrlBreaks{\do\A\do\B\do\C\do\D\do\E\do\F\do\G\do\H\do\I\do\J\do\K\do\L\do\M\do\N\do\O\do\P\do\Q\do\R\do\S\do\T\do\U\do\V\do\W\do\X\do\Y\do\Z\do\[\do\\\do\]\do\^\do\_\do\`\do\a\do\b\do\c\do\d\do\e\do\f\do\g\do\h\do\i\do\j\do\k\do\l\do\m\do\n\do\o\do\p\do\q\do\r\do\s\do\t\do\u\do\v\do\w\do\x\do\y\do\z\do\0\do\1\do\2\do\3\do\4\do\5\do\6\do\7\do\8\do\9\do\.\do\@\do\\\do\/\do\!\do\_\do\|\do\;\do\>\do\]\do\)\do\,\do\?\do\'\do+\do\=\do\#\do\-}
\usepackage[breaklinks = true]{hyperref}
\lstset{language = c++, breaklines = true, tabsize = 2, numbers = left, extendedchars = false, basicstyle = {\ttfamily \footnotesize}, keywordstyle=\color{blue!70}, commentstyle=\color{red!70}, frame=shadowbox, rulesepcolor=\color{red!20!green!20!blue!20}, numberstyle={\color[RGB]{0,192,192}}}
\mathchardef\myhyphen="2D
% switch-case environment definitions
\algblock{switch}{endswitch} 
\algblock{case}{endcase}
%\algrenewtext{endswitch}{\textbf{end switch}} %If end switch is expected to appear, uncomment this line.
\algtext*{endswitch} % Make end switch disappear
\algtext*{endcase}
\allowdisplaybreaks
\DeclareMathOperator*{\argmin}{\arg\!\min}
\DeclareMathOperator*{\argmax}{\arg\!\max}

\begin{document}
\fi
\newcommand{\angless}[2]{#1^{\left\langle #2\right\rangle}}
\newcommand{\bracketss}[2]{#1^{\left[ #2\right]}}
\chapter{Sequence Models}
Examples of sequence model use cases:
\begin{itemize}
  \item Speech recognition: audio clip $\rightarrow$ transcript
  \item Music generation: genre of music $\rightarrow$ music clip
  \item Sentiment classification: ``there is nothing to like in this movie'' $\rightarrow$ one star
  \item DNA sequence analysis: DNA sequence $\rightarrow$ sequences encoding proteins
  \item Machine translation: French sentence $\rightarrow$ English sentence
  \item Video activity recognition: video of people running $\rightarrow$ running
  \item Named entity recognition: text $\rightarrow$ names in the text
\end{itemize}
Notation (using named entity recognition as an example):
\begin{itemize}
\item Input $x$: \textit{Harry Porter} and \textit{Hermione Granger} invented a new spell. $\angless{x}{t}$ is the $t^{th}$ word.
\item Output $y$ (whether each word is part of a name): 110110000. $\angless{y}{t}$ for the $t^{th}$ word.
\item Length of the sequences $T_x=9, T_y=9$
\item Same notation as before for training example index: $X^{(i)\langle t\rangle}$, $Y^{(i)\langle t\rangle}$, $T_x^{(i)}$, $T_y^{(i)}$.
\item Content of $\angless{x}{t}$: define a vocabulary containing all words. Each word is represented by a one-hot vector (one at the index of the word in the vocabulary) whose dimension is the vocabulary size. The vocabulary contains an \textit{unknown} item to represent words not in the vocabulary.
\end{itemize}
\section{Recurrent Neural Networks}
\subsection{Basic RNN}
\subsubsection{Problems of Standard NN}
\begin{itemize}
  \item Input/output of different training examples have different dimensions.
  \item Features learned across different positions of text are not shared.
  \item Large number of parameters because input size is large.
\end{itemize}
\subsubsection{RNN Structure}
\begin{center}
  \begin{tabular}{cccccccccc}
    & $\angless{\hat{y}}{1}$ & & $\angless{\hat{y}}{2}$ & & $\angless{\hat{y}}{3}$ & & & & $\angless{\hat{y}}{T_y}$\\
    & $\uparrow$ & & $\uparrow$ & & $\uparrow$ & & & & $\uparrow$\\ 
    $\xrightarrow{\angless{a}{0}}$& NN & $\xrightarrow{\angless{a}{1}}$ & NN & $\xrightarrow{\angless{a}{2}}$ & NN & $\xrightarrow{\angless{a}{3}}$ & $\cdots$ & $\xrightarrow{\angless{a}{T_x-1}}$ & NN \\ 
    & $\uparrow$ & & $\uparrow$ & & $\uparrow$ & & & & $\uparrow$\\ 
    & $\angless{x}{1}$ & & $\angless{x}{2}$ & & $\angless{x}{3}$ & & & & $\angless{x}{T_x}$\\
  \end{tabular}
\end{center}
\begin{itemize}
  \item The activation for $\angless{x}{t}$ is fed as an input for the next item in the sequence $\angless{x}{t+1}$.\footnote{Later items in the sequence are not used for making predictions. The issue will be solved by BRNN (B for bidirectional).}
  \item Parameters are shared among time steps.
  \item $\angless{a}{0}$ is a fake time 0 activation, usually all zero.
  \item Here we assume $T_x=T_y$. 
  \end{itemize}
\subsubsection{Calculation}
\begin{align*}
\angless{a}{t}&=g_a\left(W_{aa}\angless{a}{t-1}+W_{ax}\angless{x}{t}+b_a\right)=g_a\left(W_{a}\left[\angless{a}{t-1}, \angless{x}{t}\right]+b_a\right)\\
\angless{\hat{y}}{t}&=g_y\left(W_{ya}\angless{a}{t}+b_y\right)=g_y\left(W_{y}\angless{a}{t}+b_y\right)
\end{align*}
\begin{itemize}
\item $W_a\equiv\left[W_{aa}, W_{ax}\right], W_y\equiv W_{ya}$
\item $g_a$ is usually $\tanh$ or Relu.
\item $g_y$ depends on output $\hat{y}$ (e.g. sigmoid for binary output).
\end{itemize}
\subsubsection{Backpropagation through Time}
Loss function:
\begin{align*}
  \angless{\mathcal{L}}{t}\left(\angless{\hat{y}}{t}, \angless{y}{t}\right)&=-\angless{y}{t}\log\angless{\hat{y}}{t}-\left(1-\angless{y}{t}\right)\log\left(1-\angless{\hat{y}}{t}\right)\\
  \mathcal{L}&=\displaystyle\sum_{t=1}^{T_x}\angless{\mathcal{L}}{t}\left(\angless{\hat{y}}{t}, \angless{y}{t}\right)
\end{align*}
\begin{center}
  \begin{tabular}{cccccccccc}
    & $\angless{\mathcal{L}}{1}$ & & $\angless{\mathcal{L}}{2}$ & & $\angless{\mathcal{L}}{3}$ & & & & $\angless{\mathcal{L}}{T_y}$\\
    & $\uparrow\color{red}\downarrow$ & & $\uparrow\color{red}\downarrow$ & & $\uparrow\color{red}\downarrow$ & & & & $\uparrow\color{red}\downarrow$\\ 
    & $\angless{\hat{y}}{1}$ & & $\angless{\hat{y}}{2}$ & & $\angless{\hat{y}}{3}$ & & & & $\angless{\hat{y}}{T_y}$\\
    & $\uparrow\color{red}\downarrow$ & & $\uparrow\color{red}\downarrow$ & & $\uparrow\color{red}\downarrow$ & & & & $\uparrow\color{red}\downarrow$\\ 
    \makecell{$\xrightarrow{\angless{a}{0}}$\\\color{red}$\xleftarrow{\hphantom{\angless{a}{0}}}$}& NN & \makecell{$\xrightarrow{\angless{a}{1}}$\\\color{red}$\xleftarrow{\hphantom{\angless{a}{0}}}$} & NN & \makecell{$\xrightarrow{\angless{a}{2}}$\\\color{red}$\xleftarrow{\hphantom{\angless{a}{2}}}$} & NN & \makecell{$\xrightarrow{\angless{a}{3}}$\\\color{red}$\xleftarrow{\hphantom{\angless{a}{0}}}$} & $\cdots$ & \makecell{$\xrightarrow{\angless{a}{T_x-1}}$\\\color{red}$\xleftarrow{\hphantom{\angless{a}{T_x-1}}}$} & NN \\ 
    & $\uparrow$ & & $\uparrow$ & & $\uparrow$ & & & & $\uparrow$\\ 
    & $\angless{x}{1}$ & & $\angless{x}{2}$ & & $\angless{x}{3}$ & & & & $\angless{x}{T_x}$\\
  \end{tabular}
\end{center}
\subsection{Different Types of RNN}
\subsubsection{Many-to-many}
The RNN we saw above is a many-to-many architecture, satisfying $T_x=T_y$:
\begin{center}
  \begin{tabular}{cccccccccc}
    & $\angless{\hat{y}}{1}$ & & $\angless{\hat{y}}{2}$ & & $\angless{\hat{y}}{3}$ & & & & $\angless{\hat{y}}{T_y}$\\
    & $\uparrow$ & & $\uparrow$ & & $\uparrow$ & & & & $\uparrow$\\ 
    $\xrightarrow{\angless{a}{0}}$& NN & $\xrightarrow{\angless{a}{1}}$ & NN & $\xrightarrow{\angless{a}{2}}$ & NN & $\xrightarrow{\angless{a}{3}}$ & $\cdots$ & $\xrightarrow{\angless{a}{T_x-1}}$ & NN \\ 
    & $\uparrow$ & & $\uparrow$ & & $\uparrow$ & & & & $\uparrow$\\ 
    & $\angless{x}{1}$ & & $\angless{x}{2}$ & & $\angless{x}{3}$ & & & & $\angless{x}{T_x}$\\
  \end{tabular}
\end{center}
\subsubsection{Many-to-one}
The assumption $T_x=T_y$ may not always hold. For example, sentiment classification is a many-to-one architecture:
\begin{center}
  \begin{tabular}{cccccccccc}
    & & & & & & & & & $\hat{y}$\\
    & & & & & & & & & $\uparrow$\\ 
    $\xrightarrow{\angless{a}{0}}$& NN & $\xrightarrow{\angless{a}{1}}$ & NN & $\xrightarrow{\angless{a}{2}}$ & NN & $\xrightarrow{\angless{a}{3}}$ & $\cdots$ & $\xrightarrow{\angless{a}{T_x-1}}$ & NN \\ 
    & $\uparrow$ & & $\uparrow$ & & $\uparrow$ & & & & $\uparrow$\\ 
    & $\angless{x}{1}$ & & $\angless{x}{2}$ & & $\angless{x}{3}$ & & & & $\angless{x}{T_x}$\\
  \end{tabular}
\end{center}
\subsubsection{One-to-many}
Music generation is a one-to-many architecture:
\begin{center}
  \begin{tabular}{cccccccccc}
    & $\angless{\hat{y}}{1}$ & & $\angless{\hat{y}}{2}$ & & $\angless{\hat{y}}{3}$ & & & & $\angless{\hat{y}}{T_y}$\\
    & $\uparrow$ & & $\uparrow$ & & $\uparrow$ & & & & $\uparrow$\\ 
    $\xrightarrow{\angless{a}{0}}$& NN & $\xrightarrow{\angless{a}{1}}$ & NN & $\xrightarrow{\angless{a}{2}}$ & NN & $\xrightarrow{\angless{a}{3}}$ & $\cdots$ & $\xrightarrow{\angless{a}{T_x-1}}$ & NN \\ 
    & $\uparrow$ \\ 
    & $x$ \\
  \end{tabular}
\end{center}
In this architecture, the output $\angless{y}{t}$ is often fed to the $t+1^{th}$ step as an input:
\begin{center}
  \begin{tabular}{cccccccccc}
    & $\angless{\hat{y}}{1}$ & & $\angless{\hat{y}}{2}$ & & $\angless{\hat{y}}{3}$ & & & & $\angless{\hat{y}}{T_y}$\\
    & $\uparrow$ & & $\uparrow$ & & $\uparrow$ & & & & $\uparrow$\\ 
    $\xrightarrow{\angless{a}{0}}$& NN & $\xrightarrow{\angless{a}{1}}$ & NN & $\xrightarrow{\angless{a}{2}}$ & NN & $\xrightarrow{\angless{a}{3}}$ & $\cdots$ & $\xrightarrow{\angless{a}{T_x-1}}$ & NN \\ 
    & $\uparrow$ & & $\uparrow$ & & $\uparrow$ & & & & $\uparrow$\\ 
    & $x$ & & \color{red}$\angless{\hat{y}}{1}$ & & \color{red}$\angless{\hat{y}}{2}$ & & & & \color{red}$\angless{\hat{y}}{T_y-1}$\\
  \end{tabular}
\end{center}
\subsubsection{One-to-one}
One-to-one structure is a simple standard NN(no actual RNN):
\begin{center}
  \begin{tabular}{c}
    $\angless{\hat{y}}{1}$\\
    $\uparrow$\\ 
    NN\\
    $\uparrow$\\ 
    $\angless{x}{1}$ 
  \end{tabular}
\end{center}
\subsubsection{Many-to-many(machine translation)}
Machine translation uses a special many-to-many architecture:
\begin{center}
  \begin{tabular}{cccccccc}
    $\xrightarrow{\angless{a}{0}}$& NN & $\xrightarrow{\angless{a}{1}}$ & NN & $\xrightarrow{\angless{a}{2}}$ & $\cdots$ & $\xrightarrow{\angless{a}{T_x-1}}$ & NN \\ 
    & $\uparrow$ & & $\uparrow$ & & & & $\uparrow$ \\
    & $\angless{x}{1}$ & & $\angless{x}{2}$ & & & & $\angless{x}{T_x}$ \\\arrayrulecolor{red}\hline
    & $\angless{\hat{y}}{1}$ & & $\angless{\hat{y}}{2}$ & & & & $\angless{\hat{y}}{T_y}$\\
    & $\uparrow$ & & $\uparrow$ & & & & $\uparrow$\\
    $\xrightarrow{\angless{a}{T_x}}$ & NN & $\xrightarrow{\angless{a}{T_x+1}}$ & NN & $\xrightarrow{\angless{a}{T_x+2}}$ &  $\cdots$ & $\xrightarrow{\angless{a}{T_x+T_y-1}}$ & NN \\
  \end{tabular}
\end{center}
It comprises an encoder ($1\sim T_x$) and a decoder ($T_x+1\sim T_x+T_y$).
\subsection{Language Model \& Sequence Generation}
A language model estimates the probability of a sentence:
\[P\left(\angless{y}{1},\angless{y}{2},\cdots, \angless{y}{T_y}\right)\]
For example, for speech recognition, when hearing a sentence \textit{the apple and pear salad}, a good system should output 
\[P(\textit{the apple and {\color{red}pear} salad})\gg P(\textit{the apple and {\color{red}pair} salad})\]
To build a language model using RNN, the training set is a corpus\footnote{NLP terminology. A large set.} of text. Each sentence is tokenized into a series of tokens:
\begin{center}
  \begin{tabular}{ccccccccc}
    Cats & average & 15 & hours & of & sleep & a & day & $\langle\text{EOS}\rangle$\footnotemark\\
    $\angless{y}{1}$ & $\angless{y}{2}$ & $\angless{y}{3}$ & $\angless{y}{4}$ & $\angless{y}{5}$ & $\angless{y}{6}$ & $\angless{y}{7}$ & $\angless{y}{8}$ &  $\angless{y}{9}$\\
  \end{tabular}
\end{center}
\footnotetext{EOS means end of sentence.}
If a word does not belong to the vocabulary, it is replaced with a unique token UNK, which stands for unknown words. After the tokenization, we build an RNN to model the chance of different sequences:
\begin{center}
  \begin{tabular}{cccccccc}
    & $\angless{\hat{y}}{1}$ & & $\angless{\hat{y}}{2}$ & & & & $\angless{\hat{y}}{9}$\\
    & $\uparrow$ & & $\uparrow$ & & & & $\uparrow$\\ 
    $\xrightarrow{\angless{a}{0}=\mathbf{0}}$& NN & $\xrightarrow{\angless{a}{1}}$ & NN & $\xrightarrow{\angless{a}{2}}$ & $\cdots$ & $\xrightarrow{\angless{a}{8}}$ & NN \\ 
    & $\uparrow$ & & $\uparrow$ & & & & $\uparrow$\\ 
    & $\angless{x}{1}=\mathbf{0}$ & & $\angless{x}{2}=\angless{\hat{y}}{1}$ & & & & $\angless{x}{9}=\angless{\hat{y}}{8}$\\
  \end{tabular}
\end{center}
\begin{itemize}
  \item $\angless{\hat{y}}{t}$ is a softmax output representing the conditional probability distribution of the $t^{th}$ word given the first $t-1$ words. It's a vector whose dimension is the vocabulary size.
  \[\angless{\hat{y}}{t}_i=P\left(w_i\left\vert \angless{y}{1}\angless{y}{2}\cdots \angless{y}{t-1}\right.\right)\]
  \item The input $\angless{x}{t}$ is the ${t-1}^{th}$ token $\angless{y}{t-1}$.
  \item Loss function:
  \[\mathcal{L}=\displaystyle\sum_{t}\angless{\mathcal{L}}{t}\left(\angless{\hat{y}}{t},\angless{y}{t}\right)=-\displaystyle\sum_{t}\left[\angless{y}{t}\log\angless{\hat{y}}{t}+\left(1-\angless{y}{t}\right)\log\left(1-\angless{\hat{y}}{t}\right)\right]\]
  \item After training the RNN on a training set, the obtained model can calculate the probability of a sentence:
  \begin{align*}
    P\left(\angless{y}{1},\angless{y}{2},\angless{y}{3}\right)&=P\left(\angless{y}{1}\right)\cdot P\left(\angless{y}{2}\left\vert \angless{y}{1}\right.\right)\cdot P\left(\angless{y}{3}\left\vert \angless{y}{1}\angless{y}{2}\right.\right)\\
    &=\angless{\hat{y}}{1}_{\angless{y}{1}}\cdot\angless{\hat{y}}{2}_{\angless{y}{2}}\cdot\angless{\hat{y}}{3}_{\angless{y}{3}}
  \end{align*}
  \item Character-level model: take characters, as well as punctuations and spaces, instead of words as tokens. No unknown token, but longer sequence. Not good at capturing long range dependencies.
  \item Sampling novel sequences: randomly sample a word according to the probability distribution indicated by $\angless{\hat{y}}{t}$, feed it to the next step, and repeat. This process leads to a randomly generated sequence of words.
\end{itemize}
\subsection{Solving Vanishing Gradients}
Like CNN, RNN suffers from vanishing gradients problem, making it hard for RNN to capture long range dependencies. Exploding gradients also happen, but can be solved by gradient clipping.
\subsubsection{Gated Recurrent Unit(GRU)}
\begin{itemize}
  \item Simplified GRU: 
  \begin{align*}
    \angless{\tilde{c}}{t}&=\tanh\left(W_c\left[\angless{c}{t-1},\angless{x}{t}\right]+b_c\right)\\
    \Gamma_u&=\sigma\left(W_u\left[\angless{c}{t-1},\angless{x}{t}\right]+b_u\right)\\
    \angless{c}{t}&=\Gamma_u\ast \angless{\tilde{c}}{t}+\left(1-\Gamma_u\right)\ast \angless{c}{t}
  \end{align*}
  \begin{itemize}
    \item $c$ means memory cell. For GRU, $\angless{a}{t}=\angless{c}{t}$.
    \item $\angless{\tilde{c}}{t}$ is a candidate to replace $\angless{c}{t}$. 
    \item $\Gamma_u$ is an update gate controlling whether to replace $\angless{c}{t}$ with $\angless{\tilde{c}}{t}$. For most inputs, $\Gamma_u$ (sigmoid output) is close to 1 or 0. For 1, the replacement is carried out. For 0, the value of $\angless{c}{t}$ is preserved. 
    \item GRU solves the vanishing gradients problem because it preserves the value of $\angless{c}{t}$ in a lot of steps, allowing long-range dependencies to be captured.
    \item $\angless{c}{t}, \angless{\tilde{c}}{t}, \Gamma_u$ are vectors of the same dimension. $\ast$ is element-wise multiplication.
  \end{itemize}
  \item Full GRU:
  \begin{align*}
    \angless{\tilde{c}}{t}&=\tanh\left(W_c\left[\Gamma_r\ast \angless{c}{t-1},\angless{x}{t}\right]+b_c\right)\\
    \Gamma_u&=\sigma\left(W_u\left[\angless{c}{t-1},\angless{x}{t}\right]+b_u\right)\\
    \Gamma_r&=\sigma\left(W_r\left[\angless{c}{t-1},\angless{x}{t}\right]+b_r\right)\\
    \angless{c}{t}&=\Gamma_u\ast \angless{\tilde{c}}{t}+\left(1-\Gamma_u\right)\ast \angless{c}{t}
  \end{align*}
  \begin{itemize}
    \item $\Gamma_r$ is a relevance gate controlling the relevance between $\angless{\tilde{c}}{t}$ and $\angless{c}{t-1}$.
    \item The addition of $\Gamma_r$ is a result of research practice. Researcher converged to it after various attempts of architecture. 
  \end{itemize}
\end{itemize}
\subsubsection{Long Short Term Memory(LSTM)}
\begin{align*}
  \angless{\tilde{c}}{t}&=\tanh\left(W_c\left[\angless{a}{t-1},\angless{x}{t}\right]+b_c\right)\\
  \Gamma_u&=\sigma\left(W_u\left[\angless{a}{t-1},\angless{x}{t}\right]+b_u\right)\\
  \Gamma_f&=\sigma\left(W_f\left[\angless{a}{t-1},\angless{x}{t}\right]+b_f\right)\\
  \Gamma_o&=\sigma\left(W_o\left[\angless{a}{t-1},\angless{x}{t}\right]+b_o\right)\\
  \angless{c}{t}&=\Gamma_u\ast \angless{\tilde{c}}{t}+\Gamma_f \ast \angless{c}{t}\\
  \angless{a}{t}&=\Gamma_o\ast \tanh\left(\angless{c}{t}\right)
\end{align*}
\begin{itemize}
  \item $\Gamma_u$: update gate. $\Gamma_f$: forget gate. $\Gamma_o$: output gate.
  \item Peephole connection: $\angless{c}{t}$ used to calculate gates.
\end{itemize}
\subsection{Complex RNNs}
Using the standard RNN, including GRU and LSTM as building blocks, we can build more complex RNN structures.
\subsubsection{Bidirectional RNN(BRNN)}
The RNNs above are uni-directional: no information from future steps is used. A BRNN solves the problem by also calculating activation from the reverse direction.
\newcommand{\xbidirectionalarrow}[2]{\makecell{$\xrightarrow{\angless{\overrightarrow{a}}{#1}}$\\\color{red}$\xleftarrow{{\angless{\overleftarrow{a}}{#2}}}$}}
\begin{center}
  \begin{tabular}{ccccccccc}
    & $\angless{\hat{y}}{1}$ & & $\angless{\hat{y}}{2}$ & & $\angless{\hat{y}}{3}$ & & $\angless{\hat{y}}{4}$\\
    & $\uparrow$ & & $\uparrow$ & & $\uparrow$ & & $\uparrow$\\ 
    \xbidirectionalarrow{0}{1} & NN & \xbidirectionalarrow{1}{2} & NN & \xbidirectionalarrow{2}{3} & NN & \xbidirectionalarrow{3}{4} & NN & \xbidirectionalarrow{4}{5} \\
    & $\uparrow$ & & $\uparrow$ & & $\uparrow$ & & $\uparrow$\\ 
    & $\angless{x}{1}$ & & $\angless{x}{2}$ & & $\angless{x}{3}$ & & $\angless{x}{4}$\\
  \end{tabular}
\end{center}
Now the softmax output at each step becomes:
\begin{align*}
  \angless{\hat{y}}{t}=g\left(W_y\left[\angless{\overrightarrow{a}}{t},\angless{\overleftarrow{a}}{t}\right]+b_y\right)
\end{align*}
\subsubsection{Deep RNN}
\newcommand{\drnnunit}[3]{#1^{\left[#2\right]\left\langle #3\right\rangle}}
\begin{center}
  \begin{tabular}{ccccccccc}
    && $\angless{y}{1}$ & & $\angless{y}{2}$ & & $\angless{y}{3}$ & & $\angless{y}{4}$\\
    && $\uparrow$ & &  $\uparrow$ & &  $\uparrow$ & &  $\uparrow$\\
    $\drnnunit{a}{3}{0}$ & $\rightarrow$ & $\boxed{\drnnunit{a}{3}{1}}$ & $\rightarrow$ & $\boxed{\drnnunit{a}{3}{2}}$ & $\rightarrow$ & $\boxed{\drnnunit{a}{3}{3}}$ & $\rightarrow$ & $\boxed{\drnnunit{a}{3}{4}}$ \\
    && $\uparrow$ & &  $\uparrow$ & &  $\uparrow$ & &  $\uparrow$\\
    $\drnnunit{a}{2}{0}$ & $\rightarrow$ & $\boxed{\drnnunit{a}{2}{1}}$ & $\rightarrow$ & $\boxed{\drnnunit{a}{2}{2}}$ & $\rightarrow$ & $\boxed{\drnnunit{a}{2}{3}}$ & $\rightarrow$ & $\boxed{\drnnunit{a}{2}{4}}$ \\
    && $\uparrow$ & &  $\uparrow$ & &  $\uparrow$ & &  $\uparrow$\\
    $\drnnunit{a}{1}{0}$ & $\rightarrow$ & $\boxed{\drnnunit{a}{1}{1}}$ & $\rightarrow$ & $\boxed{\drnnunit{a}{1}{2}}$ & $\rightarrow$ & $\boxed{\drnnunit{a}{1}{3}}$ & $\rightarrow$ & $\boxed{\drnnunit{a}{1}{4}}$ \\
  \end{tabular}
\end{center}
\begin{itemize}
  \item To calculate the activation:
  \begin{align*}
    \drnnunit{a}{l}{t}=g\left(\bracketss{W_a}{l}\left[\drnnunit{a}{l}{t-1},\drnnunit{a}{l-1}{t}\right]+\bracketss{b_a}{l}\right)
  \end{align*}
  \item $\bracketss{W_a}{l},\bracketss{b_a}{l}$ is the same for all hidden units in the same layer.
  \item Unlike CNN, $n_l=3$ is already deep for a DRNN.
  \item Optionally, further layers can be added along the vertical direction in the figure above (i.e. new layers without the horizontal (temporal) connections).
\end{itemize}
\ifx\PREAMBLE\undefined
\end{document}
\fi