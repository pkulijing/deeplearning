\ifx\PREAMBLE\undefined
\documentclass{report}
\usepackage[format = hang, font = bf]{caption}
\usepackage{graphicx}
\usepackage{array}
\usepackage{amsmath}
\usepackage{colortbl}
\usepackage{empheq}
\usepackage{mathtools}
\usepackage{boxedminipage}
\usepackage{listings}
\usepackage{makecell}%diagonal line in table
\usepackage{float}%allowing forceful figure[H]
\usepackage{xcolor}
\usepackage{amsfonts}%allowing \mathbb{R}
\usepackage{amssymb}
\usepackage{bm}
\usepackage{alltt}
\usepackage{amsthm}
\usepackage[super]{nth}
\usepackage{tikz}
\usetikzlibrary{positioning, calc}
\usepackage{multirow}
\usepackage{algorithmicx}
\usepackage[chapter]{algorithm} 
%chapter option ensures that algorithms are numbered within each chapter rather than in the whole article
\usepackage[noend]{algpseudocode} %If end if, end procedure, etc is expected to appear, remove the noend option
\usepackage{xspace}
\usepackage{color}
\usepackage{url}
\def\UrlBreaks{\do\A\do\B\do\C\do\D\do\E\do\F\do\G\do\H\do\I\do\J\do\K\do\L\do\M\do\N\do\O\do\P\do\Q\do\R\do\S\do\T\do\U\do\V\do\W\do\X\do\Y\do\Z\do\[\do\\\do\]\do\^\do\_\do\`\do\a\do\b\do\c\do\d\do\e\do\f\do\g\do\h\do\i\do\j\do\k\do\l\do\m\do\n\do\o\do\p\do\q\do\r\do\s\do\t\do\u\do\v\do\w\do\x\do\y\do\z\do\0\do\1\do\2\do\3\do\4\do\5\do\6\do\7\do\8\do\9\do\.\do\@\do\\\do\/\do\!\do\_\do\|\do\;\do\>\do\]\do\)\do\,\do\?\do\'\do+\do\=\do\#\do\-}
\usepackage[breaklinks = true]{hyperref}
\lstset{language = c++, breaklines = true, tabsize = 2, numbers = left, extendedchars = false, basicstyle = {\ttfamily \footnotesize}, keywordstyle=\color{blue!70}, commentstyle=\color{red!70}, frame=shadowbox, rulesepcolor=\color{red!20!green!20!blue!20}, numberstyle={\color[RGB]{0,192,192}}}
\mathchardef\myhyphen="2D
% switch-case environment definitions
\algblock{switch}{endswitch} 
\algblock{case}{endcase}
%\algrenewtext{endswitch}{\textbf{end switch}} %If end switch is expected to appear, uncomment this line.
\algtext*{endswitch} % Make end switch disappear
\algtext*{endcase}
\allowdisplaybreaks
\DeclareMathOperator*{\argmin}{\arg\!\min}
\DeclareMathOperator*{\argmax}{\arg\!\max}

\begin{document}
\fi
\chapter{Convolutional Neural Networks}
Computer vision is developping rapidly thans to deep learning. It also provides inspiration for other fields in which DL is applied. Computer vision problems include: 
\begin{itemize}
  \item image classification
  \item object detection: detect object and draw bounding boxes. One image might contain multiple objects
  \item neural style transfer: content image + style image $\rightarrow$ content image repainted
\end{itemize}
One of the biggest challenges of applying DL in CV is its big input size: for a 1000$\times$1000 image, the input size is 3M, if the 1st layer contains 1000 hidden units, then $W^{[1]}$ will have 3B elements, which makes it hard not to overfit, and also requires more hardware resources for training. The solution is the convolutional operation.
\section{Convolutional Operation}
Convolutional operation\footnote{In mathematical textbooks, the filter should be flipped ($F^f_{ij}=F_{m-1-i, n-1-j}$)before the operation above. The convolutional operation in DL is called cross-correlation in math context. The flipping makes the operation assosiative.}: with $m\times n$ matrix $A$ and $f\times f$ matrix $F$ (kernel, or filter), we define their convolution $C$ as follows:
    \[C_{ij}=\left(A*F\right)_{ij}=\displaystyle\sum_{p=0}^{f-1}\displaystyle\sum_{q=0}^{f-1}A_{i+p,j+q}*F_{pq}, \left(0\le i\le m-f, 0\le j\le n-f\right)\]
obviously $C$ is a $m-f+1\times n-f+1$ matrix.
\subsection{Edge Detection}
\begin{itemize}
  \item Vertical edge detection kernel:
  $F_{ve}=\begin{bmatrix}
    1 & 0 & -1 \\
    1 & 0 & -1 \\
    1 & 0 & -1
  \end{bmatrix}$. Intuition: 
  \[\begin{bmatrix}
    10 & 10 & 10 & 0 & 0 & 0 \\
    10 & 10 & 10 & 0 & 0 & 0 \\
    10 & 10 & 10 & 0 & 0 & 0 \\
    10 & 10 & 10 & 0 & 0 & 0 \\
    10 & 10 & 10 & 0 & 0 & 0 \\
    10 & 10 & 10 & 0 & 0 & 0
  \end{bmatrix} * \begin{bmatrix}
    1 & 0 & -1 \\
    1 & 0 & -1 \\
    1 & 0 & -1
  \end{bmatrix}=\begin{bmatrix}
    0 & 30 & 30 & 0 \\
    0 & 30 & 30 & 0 \\
    0 & 30 & 30 & 0 \\
    0 & 30 & 30 & 0
  \end{bmatrix} \]
  \item When the detected edge contains positive values (30 above), the edge has bright pixels on the left and dark pixels on the right, vice versa.
  \item $F_{he}=F_{ve}^{\mathsf{T}}$ is a horizontal edge detection kernel.
  \item Instead of hand-designing the value of the edge detection kernel, these parameters can be learned. 
\end{itemize}
\subsection{Padding}
\begin{itemize}
  \item Downsides of the convolutional opertion above:
  \begin{itemize}
    \item Image shrinks after convolution operations.
    \item Corner pixels are used only once; edge pixels are used only 2-3 times; etc
  \end{itemize} 
  \item Solution: padding around the image border ($m\times n\rightarrow m+1\times n+1$). The value of the padded pixels is usually 0. The number of rows / columns to pad is called padding amound $p$.
  \item Valid padding: no padding. image shrinks by $f-1$ pixels along both sides for an $f\times f$ filter. 
  \item Same padding: pad around the border so that image size remains the same.
  \[n+2p-f+1=n\Rightarrow p=\frac{f-1}{2}\]
  In CV, $f$ is usually odd, so that no asymmetric padding is needed, and there exists a ``center'' position of the filter.
\end{itemize}
\subsection{Strided Convolutions}
Strided convolutional operation: with $m\times n$ matrix $A$ and $f\times f$ matrix $F$ (kernel, or filter), we define their strided convolution $C$ with stride $s$ as follows:
\begin{align*}
  C_{ij}&=\left(A*_sF\right)_{ij}=\displaystyle\sum_{p=0}^{f-1}\displaystyle\sum_{q=0}^{f-1}A_{s*i+p,s*j+q}*F_{pq}\\
  0&\le i\le \left\lfloor\frac{m-f}{s}\right\rfloor, 0\le j\le \left\lfloor\frac{n-f}{s}\right\rfloor
\end{align*}
Considering padding, $C$ is a $\left\lfloor\frac{m+2p-f}{s}\right\rfloor+1\times\left\lfloor\frac{n+2p-f}{s}\right\rfloor+1$ matrix.
\subsection{Convolution Over Volumes}
\begin{itemize}
  \item An image is usually more than a 2D matrix. An RGB image has 3 channels, while an RGBA image has 4 channels. 
  \item The number of channels $n_c$ (also called depth) calls for the addition of another dimension to both the image and the filter.
  \begin{align*}
    C_{ij}&=\left(A*F\right)_{ij}=\displaystyle\sum_{p=0}^{f-1}\displaystyle\sum_{q=0}^{f-1}\displaystyle\sum_{r=0}^{n_c-1}A_{i+p,j+q,k}*F_{pqk}\\
    0&\le i\le m-f, 0\le j\le n-f
  \end{align*}
  Note that the output is still a 2D matrix.
  \item Multiple filters can be applied at the same time so that the output is also a volume. e.g. this can be used to detect edges along both vertical and horizontal directions.
\end{itemize}
\section{One Layer of CNN}
\ifx\PREAMBLE\undefined
\end{document}
\fi