\ifx\PREAMBLE\undefined
\documentclass{report}
\usepackage[format = hang, font = bf]{caption}
\usepackage{graphicx}
\usepackage{array}
\usepackage{amsmath}
\usepackage{mathtools}
\usepackage{boxedminipage}
\usepackage{listings}
\usepackage{makecell}%diagonal line in table
\usepackage{float}%allowing forceful figure[H]
\usepackage{xcolor}
\usepackage{amsfonts}%allowing \mathbb{R}
\usepackage{alltt}
\usepackage{algorithmicx}
\usepackage[chapter]{algorithm} 
%chapter option ensures that algorithms are numbered within each chapter rather than in the whole article
\usepackage[noend]{algpseudocode} %If end if, end procdeure, etc is expected to appear, remove the noend option
\usepackage{xspace}
\usepackage{color}
\usepackage{url}
\def\UrlBreaks{\do\A\do\B\do\C\do\D\do\E\do\F\do\G\do\H\do\I\do\J\do\K\do\L\do\M\do\N\do\O\do\P\do\Q\do\R\do\S\do\T\do\U\do\V\do\W\do\X\do\Y\do\Z\do\[\do\\\do\]\do\^\do\_\do\`\do\a\do\b\do\c\do\d\do\e\do\f\do\g\do\h\do\i\do\j\do\k\do\l\do\m\do\n\do\o\do\p\do\q\do\r\do\s\do\t\do\u\do\v\do\w\do\x\do\y\do\z\do\0\do\1\do\2\do\3\do\4\do\5\do\6\do\7\do\8\do\9\do\.\do\@\do\\\do\/\do\!\do\_\do\|\do\;\do\>\do\]\do\)\do\,\do\?\do\'\do+\do\=\do\#\do\-}
\usepackage[breaklinks = true]{hyperref}
\lstset{language = c++, breaklines = true, tabsize = 2, numbers = left, extendedchars = false, basicstyle = {\ttfamily \footnotesize}, keywordstyle=\color{blue!70}, commentstyle=\color{red!70}, frame=shadowbox, rulesepcolor=\color{red!20!green!20!blue!20}, numberstyle={\color[RGB]{0,192,192}}}
\mathchardef\myhyphen="2D
% switch-case environment definitions
\algblock{switch}{endswitch} 
\algblock{case}{endcase}
%\algrenewtext{endswitch}{\textbf{end switch}} %If end switch is expected to appear, uncomment this line.
\algtext*{endswitch} % Make end switch disappear
\algtext*{endcase}
\allowdisplaybreaks
\begin{document}
\fi
\chapter{Structuring Machine Learning Projects}
\section{Introduction to ML Strategy}
A lot of options are available when it comes to improving the performance of a ML model:
\begin{itemize}
  \item Collect more data
  \item Collect more diverse training set
  \item Train algorithm longer with gradient descent
  \item Try Adam instead of gradient descent
  \item Add L2 / dropout regularization
  \item Change network architecture: activation function, \# hidden units, bigger / smaller network
\end{itemize}
Machine Learning strategy helps to find the most promising direction to proceed.
\subsection{Orthogonalization}
Chain of assumptions in ML:
\begin{enumerate}
  \item Fit training set well on cost function (human level performance): bigger network; better optimization algorithm
  \item Fit dev set well on cost function: regularization; bigger train set
  \item Fit test set well on cost function: bigger dev set
  \item Perform well in real world: change dev set; change cost function
\end{enumerate}
If the model's performance is not ideal in one stage, there exists specific approaches to take for better performance of that stage.

Early stopping is a strategy against the orthogonalization principle: it trys to ameliorate the performance on both the train set and dev set at the same time.
\subsection{Goal Setup}
\subsubsection{Single Number Evaluation Metric}
Use appropriate metrics that can be expressed by a single number for easier tuning.

Precision $P$ \& recall $R$ $\rightarrow$ F1 score = $\frac{2}{1/P+1/R}=\frac{2PR}{P+R}$
\subsubsection{Satisficing \& Optimizing Metrics}
\begin{itemize}
  \item optimizing metric: a metric to optimize as well as possible
  \item satisficing metric: a metric that is acceptable as long as some limit is satisfied
\end{itemize}
e.g. 
\begin{itemize}
  \item For cat recognization: accuracy is optimizing metric, while running time is satisficing metric (as long as $<100ms$, it is acceptable)
  \item Waking word of smart speaker: accuracy is optimizing metric, while false positive frequency is satisficing metric (at most 1 each day).
\end{itemize}
\subsubsection{Train/Dev/Test Set Setup}
\begin{itemize}
  \item Distribution: choose dev \& test set to reflect real world data. Make sure they come from the same distribution.
  \item Size: test set should be big enough to give high confidence in the overall performance of the system. 70/30 or 60/20/20 for small dataset, 98/1/1 for big dataset.
\end{itemize}
\subsubsection{When to Change}
If doing well with metric + dev/set data does not lead to good real world performance, consider to change the metric and/or dev/test set.
\ifx\PREAMBLE\undefined
\end{document}
\fi